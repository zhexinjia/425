\documentclass[12pt]{article}
\usepackage{amsmath}
\usepackage{tipa}
\usepackage{changepage}
\usepackage{lipsum}

\begin{document}
\title{CIS 425 Assignment 2}
\author{Zhexin Jia}
\maketitle

\noindent 1. Example of a variable occurs both bound and free.
\begin{center}
$\lambda x.(\lambda y.xy)y$
\end{center}
In above lambda expression, the variable y occurs both bound and free. The first y is the binding occurrence, the second y is bound because it is in the scope of $\lambda y$, the second $y$ is free because it is not in the scope of $\lambda y$.\\
\vspace{1mm}\\
\noindent 2. Give the free variables and bound variables of there terms:\\
(a). $\lambda x.(\lambda y.xy)y$\\
Bound variable: x, y\\
Free variable: y. \\
The second x is in the scope of $\lambda x$, the second y is in the scope of $\lambda y$. So they are bound variables.\\
The third y is neither in the scope of $\lambda y$ or $\lambda x$, it is free variable.\\
\vspace{1mm}\\
(b). $\lambda k.k(\lambda f.hf)(qf)$\\
Bound variable: k, f\\
Free variable: f\\
The variable k is in the scope of $\lambda k$, the second f is in the scope of $\lambda f$, so they are bound variable.\\
The third variable f is not in the scope of $\lambda f$, it is free variable. So f is both free and bound variable.\\
\vspace{1mm}\\
3. Give the result of performing the following substitutions:\\
a.$[(\lambda y.xy)/x](x(\lambda x.yx))$\\
= $[(\lambda y.xy)/x](x(\lambda z.yz))$\\
= $((\lambda y.xy)(\lambda z.yz))$\\
= $x(\lambda z.yz)$\\
The "x" in $\lambda x.yx$ is a bounded variable to $\lambda x$(inside expression), so I rename it to "z" first before doing the outside expression substitution.\\
\vspace{1mm}\\
b. $[(\lambda x.xy)/x](\lambda y.x(\lambda x.x))$\\
= $[(\lambda x.xy)/x](\lambda f.x(\lambda z.z))$\\
= $(\lambda f.(\lambda x.xy)(\lambda z.z))$\\
= $(\lambda f.((\lambda z.z)y))$\\
= $(\lambda f.y)$\\
\vspace{1mm}\\
4. Verify following by applying $\beta$-axiom:\\
a. $SKII = I$, where $S$ is $\lambda x.\lambda y.\lambda z.(xz)(yz), K$ is $\lambda x.\lambda y.x$ and $I$ is $\lambda x.x$\\
SKII = $(\lambda x.\lambda y.\lambda z.(xz)(yz))(\lambda x.\lambda y.x)II$\\
= $(\lambda y.\lambda z.((\lambda x.\lambda y.x)z)(yz))II$\\
= $(\lambda y.\lambda z.(\lambda y.z)(yz))II$\\
= $(\lambda y.\lambda z.(\lambda y.z)(yz))(\lambda x.x)I$\\
= $(\lambda z.(\lambda y.z)((\lambda x.x)z))I$\\
= $(\lambda z.(\lambda y.z)(z))I$\\
= $(\lambda z.(\lambda y.z)(z))(\lambda x.x)$\\
= $(\lambda y.(\lambda x.x))(\lambda x.x)$\\
= $(\lambda x.x)$\\
= I\\
\vspace{1mm}\\
b.$(\lambda x.xx)II = I$\\
$(\lambda x.xx)II = (II)I$\\
= $(\lambda x.xI)I$\\
= $II$\\
= $(\lambda x.x)(\lambda x.x)$\\
= $(\lambda x.x)$\\
= $I$\\
\vspace{1mm}\\
5. $Succ = \lambda u.\lambda x.\lambda y.x(uxy)$, show $Succ(\lambda x.\lambda y.x^{n}y) = (\lambda x.\lambda y. x^{n+1}y)$\\
$Succ(\lambda x.\lambda y.x^{n}y) = (\lambda u.\lambda x.\lambda y.x(uxy))(\lambda x.\lambda y.x^{n}y)$\\
= $(\lambda u.\lambda z.\lambda f.z(uzf))(\lambda x.\lambda y.x^{n}y)$\\
= $(\lambda z.\lambda f.z((\lambda x.\lambda y.x^{n}y)zf))$\\
= $(\lambda z.\lambda f.z((\lambda y.z^{n}y)f))$\\
= $(\lambda z.\lambda f.z(z^{n}f))$\\
= $(\lambda z.\lambda f.z^{n+1}f)$\\
= $(\lambda x.\lambda y.x^{n+1}y)$\\
\vspace{1mm}\\
6. Definition of f into lambda calculus: 
\indent $\lambda f.\lambda g.g$\\
$f(f) = (\lambda f.\lambda g.g)(\lambda f.\lambda g.g)$\\
= $(\lambda x.\lambda y.y)(\lambda f.\lambda g.g)$\\
= $(\lambda y.y)$\\
\vspace{1mm}\\
7. (a) Explain how to use map and reduce to compute the sum of first five squares, in one line.\\
$reduce(function(x,y)\{return (x+y)\}, map(function(x)\{return  (x^{2})\}, [1,2,3,4,5]))$;\\
\vspace{1mm}\\
(b) Explain how to use map and reduce to count the number of possitve numbers in an array of numbers.\\
Make a function check(x), it returns 1 if $x > 0$; otherwise returns 0. First run map(check, array) to get a new array contains 1 and 0 only, each 1 represent 1 positive number. Then run the reduce function with two parameters, the first parameter is the sum function and the second parameter is the new array we got from running the map function.\\
\textbf{ function check(x)\{\\
\indent	if (x$>$0) return 1;\\
\indent	return 0;\\
\}\\
reduce(function(x,y)\{return (x+y)\}, map(check, array))}\\
\vspace{1mm}\\
(c) Explain how to use map and/or reduce to "flatten" an array of arrays of numbers, such as [[1,2],[3,4],[5,6],[7,8,9]], to an array of numbers.\\
First make a function append(array1, array2), it use built-in JavaScript concatenation function to return the concatenation of array1 and array2. Then run the reduce function with paramters append function and orginal array.
\textbf{ function append(array1, array2)\{\\
\indent return array1.concat(array2);\\
\}\\
reduce(append, arrayOfArrays);}\\
\vspace{1mm}\\
8. (a) What is the value of g(f) in the first code example?\\
15.   \indent g(f) = f(7) = (x+y-2) = (10+7)-2 = 15;\\
\vspace{1mm}\\
(b) The call g(f) in the first code example causes the expression (x+y)-2 to be evaluated. What are the values of x and y that are used to produce the value you gave in part(a)?\\
x = 10, y = 7\\
\vspace{1mm}\\
(c)the function g(h) returns h(x), and x is the local(bound) variable 7, so the function g(h) always return h(7) no mattar what number we assigned to the global x; x = 7 and x = 10 are two different variables. x = 7 is the local(bound) variable, x = 10 is the global(free) variable\\
We are looking for the result of g(f), as explained above, g(f) always returns f(7). So 7 is the parameter of f(y) function, the local variable y is assigned to 7.\\
\vspace{1mm}\\
(d)global(free) variable x is first assigned to 5, then assigned to 10. Then we run the g(f) function. g(f) returns f(7), f(7) returns (x+y)-2, x is free variable, so we use 10 as the value of x. The reason we are not using x = 7 is because x = 7 can only be used inside the g(h) function. After we assigned 7 to the parameter of h(x), the local variable x=7 can no longer be used.\\
\vspace{1mm}\\
(e) What's the value of g(f) in the second code example?\\
10. \indent\indent(x+y) - 2 = (5+y) - 2 = (5 + 7) - 2 = 10;\\
\vspace{1mm}\\
(f) The call g(f) in the second code example cause the expression (x+y)-2 to be evaluated. What are the values of x and y that are used to produce the value you gave in part(e)?\\
x = 5, y = 7;\\
\vspace{1mm}\\
(g) Explain how the value of y is set in the sequence of calls that occur before (x+y)-2 is evaluated.\\
Bottom functions are inside of the top functions, so the execuation priority is decreasing from top to down. which menas x=5 is executed first, then f(y), then g(h), then x=10 and g(f);\\
After we assigned x = 5, we execute f(y) first although we don't know y's value, the return value of f(y) becomes (5+y) -2. Then we execute the inner function g(h), it assigned 7 to h(x) function, the return value of g(h) is h(7). At last function, we assigned x = 10, and execute g(f); so replace h(7) to f(7), then replace f(7) to f(y). so the value of y is 7 before (x+y)-2 is evaluated.\\
\vspace{1mm}\\
(h) Explain why x has the value you gave in part(f) when (x+y)-2 is evaluated.\\
Same as above, the sequence of execution is first step x = 5; second step f(y), third step g(h), fourth step x= 10 and g(f), when x = 10 is assigned and g(f) is being executed, the equation (x+y)-2 already become (5+y)-2, the x value is no longer needed for the equation. So the value of x remains 5.
























\end{document}

\end{document}